% !TeX program = xelatex
\documentclass[12pt,openright,twoside]{book}

% ===== 编码与语言 =====
\usepackage{fontspec}
\usepackage{xeCJK}
\usepackage{polyglossia}
\setdefaultlanguage{english}
% \setotherlanguage{Chinese} % 不要写 chinese 小写, polyglossia 推荐首字母大写

% ===== 字体(macOS 友好;跨平台可改 Noto)=====
\setmainfont{Times New Roman}
\setsansfont{Helvetica Neue}
\setmonofont{Menlo}
\setCJKmainfont{PingFang SC}[AutoFakeSlant=.2]
\setCJKsansfont{PingFang SC}[AutoFakeSlant=.2]
\setCJKmonofont{PingFang SC}[AutoFakeSlant=.2]

% ===== 版面与微排 =====
\usepackage[a4paper,margin=1in,headsep=18pt]{geometry}
\usepackage{setspace}
\onehalfspacing
\usepackage{microtype}

% ===== 颜色 & 超链接 =====
\usepackage[svgnames]{xcolor}
\usepackage{hyperref}
\hypersetup{
  colorlinks=true,
  linkcolor=NavyBlue,
  citecolor=NavyBlue,
  urlcolor=NavyBlue,
  pdfauthor={Translator},
  pdftitle={Arbitrage Theory in Continuous Time — CN Translation}
}

% ===== 数学环境 =====
\usepackage{amsmath,amssymb,mathtools}
\numberwithin{equation}{chapter}

\usepackage[capitalise,nameinlink,noabbrev]{cleveref}
% 常用算子与符号
\DeclareMathOperator{\tr}{tr}
\DeclareMathOperator{\diag}{diag}
\DeclareMathOperator{\Var}{Var}
\DeclareMathOperator{\Cov}{Cov}
\newcommand{\E}{\mathbb{E}}
\newcommand{\Pbb}{\mathbb{P}}
\newcommand{\Qbb}{\mathbb{Q}}
\newcommand{\F}{\mathcal{F}}
\newcommand{\R}{\mathbb{R}}
\newcommand{\1}{\mathbf{1}}
\newcommand{\dd}{\,\mathrm{d}}
\newcommand{\Ito}{It\^{o}}

% ===== 定理类(ntheorem 版)=====
\usepackage[amsmath,thmmarks,framed]{ntheorem}
\theoremstyle{plain}
\theorembodyfont{\itshape}
\newtheorem{theorem}{Theorem}[chapter]
\newtheorem{proposition}[theorem]{Proposition}
\newtheorem{lemma}[theorem]{Lemma}
\newtheorem{corollary}[theorem]{Corollary}

\theorembodyfont{\normalfont}
\newtheorem{definition}[theorem]{Definition}
\newtheorem{assumption}[theorem]{Assumption}
\newtheorem{remark}[theorem]{Remark}
\newtheorem{example}[theorem]{Example}
\newtheorem{exercise}{Exercise}[chapter]

% 无编号 proof(带方块结束符)
\theoremstyle{nonumberplain}
\theoremsymbol{\ensuremath{\square}}
\newtheorem{proof}{Proof}

% ===== 图表 =====
\usepackage{graphicx}
\usepackage{booktabs}
\usepackage{caption}
\usepackage{subcaption}
\usepackage{siunitx}
\sisetup{detect-all}

% ===== 参考文献(biber)=====
\usepackage[backend=biber,style=authoryear,sorting=nyt,maxcitenames=2,maxbibnames=10,autolang=other]{biblatex}
\addbibresource{bjork.bib}
\DeclareLanguageMapping{chinese}{english}
\DeclareLanguageMapping{Chinese}{english}

% ===== 索引与术语表 =====
\usepackage{imakeidx}
\makeindex[title=Index,columns=2,intoc]

\usepackage[toc,nonumberlist,acronym]{glossaries}
\makeglossaries
\newglossaryentry{wiener}{name={Wiener process},description={Standard Brownian motion}}
\newacronym{hjm}{HJM}{Heath--Jarrow--Morton}

% ===== 章标题与页眉页脚 =====
\usepackage{titlesec}
\titleformat{\chapter}{\Huge\bfseries}{\thechapter}{1em}{}
\titlespacing*{\chapter}{0pt}{1ex plus .2ex}{2ex}

\usepackage{fancyhdr}
\setlength{\headheight}{15pt}
\pagestyle{fancy}
\fancyhf{}
\fancyhead[RO,LE]{\thepage}
\fancyhead[RE]{\nouppercase{\leftmark}}
\fancyhead[LO]{\nouppercase{\rightmark}}

% ===== 自定义:章末 Exercises / Notes =====
\newenvironment{chapterexercises}{
  \section*{Exercises}\addcontentsline{toc}{section}{Exercises}
}{}
\newenvironment{chapternotes}{
  \section*{Notes}\addcontentsline{toc}{section}{Notes}
}{}

% 可选:原文与译文并置时的辅助命令(审校阶段用)
\newcommand{\orig}[1]{\begin{quote}\small\itshape #1\end{quote}}
\newcommand{\trans}[1]{#1}

% ===== 正文 =====
\begin{document}

\frontmatter
\begin{titlepage}
\centering
{\Huge\bfseries Arbitrage Theory in Continuous Time\\[6pt]
(连续时间套利理论·中文直译)\\[16pt]}
{\large Tomas Björk \\[8pt]}
{\large \textit{Translation Working Template}}\\[24pt]
{\large Compiled with XeLaTeX}
\vfill
{\large \today}
\end{titlepage}

\tableofcontents

\chapter*{Preface(译者说明)}
本稿作为《Arbitrage Theory in Continuous Time》的中文直译模板. 为保证可编译性, 模板移除了所有非 TeX 符号占位, 并提供稳定的定理、交叉引用、参考文献、索引与术语表配置. 后续请直接把直译内容填入对应章节. 

\mainmatter
\setcounter{chapter}{5}
% ========== Chapter 1 ==========
\chapter{投资组合的动态}
\section{引言}

我们考虑一个金融市场, 该市场由不同的资产组成, 例如股票、不同期限的债券, 或者各种金融衍生品. 在本章中, 我们将给定这些资产的价格动态, 主要目标是推导所谓的\textbf{自融资投资组合}(self-financing portfolio)的(价值)动态. 

在连续时间下, 这个问题相当复杂, 因此我们从离散时间模型开始研究. 随后, 我们将让时间步长 $\Delta t$ 逐渐趋于零, 从而得到连续时间的类比. 需要强调的是, 本节内容仅为动机与启发性的说明, 形式化的定义与理论将在下一节中给出. 

设我们研究一个金融市场, 其中时间被划分为长度为 $\Delta t$ 的周期, 并且仅在离散时刻 $n\Delta t$($n=0,1,\ldots$)进行交易. 我们考虑固定周期 $[t,\,t+\Delta t]$. 这个周期(当然 $t = n\Delta t$ 对某个 $n$ 成立)以后称为“\textbf{第 $t$ 期}”. 在接下来的内容中, 我们假设所有资产都是股票, 这仅是为了语言表述上的方便. 

\vspace{1em}
\noindent\textbf{定义 6.1}~
\begin{itemize}
    \item $N$:不同类型股票的数量. 
    \item $h_i(t)$:在区间 $[t,\,t+\Delta t]$ 内持有的第 $i$ 种股票的股数. 
    \item $h(t)$:投资组合 $[h_1(t),\ldots,h_N(t)]$, 表示在第 $t$ 期持有的头寸. 
    \item $c(t)$:在区间 $[t,\,t+\Delta t]$ 内每单位时间的消费支出额. 
    \item $S_i(t)$:在第 $t$ 期时第 $i$ 种股票的单价. 
    \item $V(t)$:第 $t$ 期时投资组合的价值. 
\end{itemize}

模型中的信息与决策结构如下:
\begin{itemize}
    \item 在时刻 $t$(即第 $t$ 期的\textbf{开始}), 我们从上一期 $t-\Delta t$ 带来一个“旧”投资组合 $h(t-\Delta t)=\{h_i(t-\Delta t),\,i=1,\ldots,N\}$. 
    \item 在时刻 $t$, 我们可以观察到价格向量 $S(t) = (S_1(t), \ldots, S_N(t))$. 
    \item 在时刻 $t$, 在观察到 $S(t)$ 之后, 我们选择一个新的投资组合 $h(t)$, 该组合将在第 $t$ 期内持有. 同时我们选择该期的消费率 $c(t)$. 二者 $h(t)$ 和 $c(t)$ 在该期内假定保持不变. 
\end{itemize}

\vspace{1em}
\noindent\textbf{注记 6.1.1}~
注意, 到目前为止, 我们只考虑\textbf{不支付股息的资产}. 带股息资产的情形稍微复杂一些, 由于它只会在第 16 章中使用, 我们在此略去其讨论. 详见第 6.3 节. 

\vspace{0.5em}
我们仅考虑所谓的\textbf{自融资投资组合–消费对} $(h,\,c)$, 即没有外生资金注入或提取的投资组合(当然不包括 $c$ 项的消费). 换句话说, 新的投资组合购买以及所有消费支出, 必须完全通过出售投资组合中原有资产来融资. 

为开始分析, 我们注意到财富 $V(t)$, 即第 $t$ 期初的财富, 等于旧投资组合 $h(t-\Delta t)$ 按当前价格的价值. 因此有
\begin{equation}
V(t) = \sum_{i=1}^{N} h_i(t-\Delta t) S_i(t)
      = h(t-\Delta t) S(t),
\tag{6.1}
\end{equation}
其中我们使用了记号
\[
xy = \sum_{i=1}^{N} x_i y_i
\]
来表示 $\mathbb{R}^N$ 中的内积. 

式 (6.1) 表明:在第 $t$ 期开始时, 我们的财富等于若将旧投资组合按今日价格全部卖出所得的金额. 我们可以将这笔资金用于两个目的:
\begin{itemize}
    \item 重新投资于新的投资组合 $h(t)$;
    \item 在第 $t$ 期内以速率 $c(t)$ 进行消费. 
\end{itemize}

新的投资组合 $h(t)$(必须以今日价格买入)的成本为
\[
\sum_{i=1}^{N} h_i(t) S_i(t) = h(t) S(t),
\]
而消费速率 $c(t)$ 的成本为 $c(t)\Delta t$. 因此, 第 $t$ 期的预算方程可写为
\begin{equation}
h(t-\Delta t) S(t) = h(t) S(t) + c(t)\Delta t.
\tag{6.2}
\end{equation}

若我们引入记号
\[
\Delta X(t) = X(t) - X(t-\Delta t),
\]
其中 $X$ 为任意过程, 则预算方程 (6.2) 可改写为
\begin{equation}
S(t)\Delta h(t) + c(t)\Delta t = 0.
\tag{6.3}
\end{equation}

由于我们的目标是得到连续时间下的预算方程, 因此自然会想到令 $\Delta t \to 0$, 将式 (6.3) 写成形式化的表达式
\[
S(t)\,dh(t) + c(t)\,dt = 0.
\]
然而, 这种做法实际上是\textbf{错误的}, 理解其中的原因非常重要. 原因如下:

\begin{itemize}
    \item 所有随机微分都必须按伊藤(Itô)意义进行解释;
    \item 伊藤积分 $\int g(t)\,dW(t)$ 被定义为如下和式的极限:
    \[
    \sum g(t_n)\,[W(t_{n+1}) - W(t_n)] ,
    \]
    其中关键在于 $W$ 的增量是\textbf{向前差分}(forward differences);
    \item 而在式 (6.3) 中, 我们使用的是\textbf{向后差分}(backward difference). 
\end{itemize}

为了得到伊藤型微分形式, 我们必须重新表述式 (6.3). 具体做法是:在左侧同时加上并减去项 $S(t-\Delta t)\Delta h(t)$, 于是预算方程变为
\begin{equation}
S(t-\Delta t)\Delta h(t) + \Delta S(t)\Delta h(t) + c(t)\Delta t = 0.
\tag{6.4}
\end{equation}

现在, 我们终于可以令 $\Delta t \to 0$, 从式 (6.4) 得到
\begin{equation}
S(t)\,dh(t) + dh(t)\,dS(t) + c(t)\,dt = 0.
\tag{6.5}
\end{equation}

再者, 令 $\Delta t \to 0$ 代入式 (6.1), 有
\begin{equation}
V(t) = h(t)S(t),
\tag{6.6}
\end{equation}
若对该式取伊藤微分, 则得到
\begin{equation}
dV(t) = h(t)\,dS(t) + S(t)\,dh(t) + dS(t)\,dh(t).
\tag{6.7}
\end{equation}

综上, 式 (6.7) 是描述任意投资组合动态的一般方程, 而式 (6.5) 是所有自融资投资组合所满足的预算方程. 将 (6.5) 代入 (6.7), 即可得到我们所需的结果:即自融资投资组合(财富)$V$ 的动态为
\begin{equation}
dV(t) = h(t)\,dS(t) - c(t)\,dt.
\tag{6.8}
\end{equation}

特别地, 在不存在消费的情形下, 财富动态简化为
\begin{equation}
dV(t) = h(t)\,dS(t).
\tag{6.9}
\end{equation}

\paragraph{注记 6.1.2}~
式 (6.9) 的自然经济解释是:在没有外生收入的模型中, 财富的所有变化都源于资产价格的变化. 因此, 式 (6.8) 和 (6.9) 似乎是显而易见的, 有人可能认为我们的推导显得多余. 事实并非如此——
当我们回忆起式 (6.8) 与 (6.9) 中的随机微分是按伊藤意义解释时, 就会意识到这一点. 特别重要的是, 在伊藤意义下, 积分变量的增量 $dS(t)$ 是\textbf{向前差分}. 如果我们选择用其他方式定义随机积分(例如采用\textbf{向后差分}, 事实上这是可行的), 则式 (6.8)–(6.9) 的形式外观将完全不同. 然而, \textbf{其本质内容}则保持不变. 

\section{自融资投资组合}

在完成上一节的推导之后, 我们自然会提出以下问题:
\begin{enumerate}
    \item 当我们令 $\Delta t \to 0$ 时, 该极限过程应在何种意义下解释?例如 $L^2$、$P$-a.s. 等?
    \item 式 (6.8) 被认为描述了连续时间下自融资投资组合的动态, 但在“现实中”, 所谓“连续时间交易”究竟意味着什么?
\end{enumerate}

这些问题的答案在于:前面的推理仅具有启发性. 现在我们给出一个纯粹数学意义下的\textbf{定义}. 各概念的解释当然与前文一致. 

\medskip
\noindent\textbf{定义 6.2}~
设给定 $N$ 维价格过程 $\{S(t);\,t \ge 0\}$. 
\begin{enumerate}
    \item \textbf{投资组合策略}(通常简称为投资组合)是任意 $\mathcal{F}^S_t$ 适应的 $N$ 维过程 $\{h(t);\,t \ge 0\}$. 
    \item 若投资组合 $h$ 具有形式
    \[
    h(t) = h(t,\,S(t)),
    \]
    其中 $h : \mathbb{R}_+ \times \mathbb{R}^N \to \mathbb{R}^N$, 则称 $h$ 为\textbf{马尔可夫型}. 
    \item 与投资组合 $h$ 对应的\textbf{价值过程} $V^h$ 定义为
    \begin{equation}
    V^h(t) = \sum_{i=1}^{N} h_i(t)\,S_i(t).
    \tag{6.10}
    \end{equation}
    \item \textbf{消费过程}是任意 $\mathcal{F}^S_t$ 适应的一维过程 $\{c(t);\,t \ge 0\}$. 
    \item 投资组合–消费对 $(h,\,c)$ 称为\textbf{自融资}, 若其价值过程 $V^h$ 满足条件
    \begin{equation}
    dV^h(t) = \sum_{i=1}^{N} h_i(t)\,dS_i(t) - c(t)\,dt
    \tag{6.11}
    \end{equation}
\end{enumerate}

即当且仅当
\[
dV^h(t) = h(t)\,dS(t) - c(t)\,dt.
\]

\paragraph{注记 6.2.1}~
一般而言, 投资组合 $h(t)$ 可以依赖于整个过去的价格轨迹 $\{S(u);\,u \le t\}$. 在接下来的内容中, 我们几乎只讨论\textbf{马尔可夫型投资组合}, 即其在时刻 $t$ 的取值仅依赖于当前时点 $t$ 与当前的价格向量 $S(t)$. 

在实际计算中, 通常更方便以相对形式而非绝对形式来描述投资组合. 换言之, 与其指定持有每种股票的绝对股数, 不如指定投资组合总价值中投资于该股票的\textbf{相对比例}. 

\medskip
\noindent\textbf{定义 6.3}~
给定投资组合 $h$, 其对应的\textbf{相对投资组合} $u$ 定义为
\begin{equation}
u_i(t) = \frac{h_i(t)S_i(t)}{V^h(t)}, \quad i = 1, \ldots, N,
\tag{6.12}
\end{equation}
其中满足
\[
\sum_{i=1}^{N} u_i(t) = 1.
\]

自融资条件可以很容易地用相对投资组合的形式改写如下. 

\medskip
\noindent\textbf{引理 6.4}~
当且仅当
\begin{equation}
dV^h(t) = V^h(t) \sum_{i=1}^{N} u_i(t)\frac{dS_i(t)}{S_i(t)} - c(t)\,dt,
\tag{6.13}
\end{equation}
投资组合–消费对 $(h,\,c)$ 是自融资的. 

\medskip
今后我们将需要以下稍具技术性的结果. 它大致说明:如果某个过程在形式上“看起来像”是自融资投资组合的价值过程, 那么它实际上确实是这样的价值过程. 

\medskip
\noindent\textbf{引理 6.5}~
设 $c$ 为消费过程, 假设存在一个标量过程 $Z$ 与向量过程 $q = (q_1, \ldots, q_N)$, 使得
\begin{equation}
dZ(t) = Z(t) \sum_{i=1}^{N} q_i(t)\frac{dS_i(t)}{S_i(t)} - c(t)\,dt,
\tag{6.14}
\end{equation}
并且
\begin{equation}
\sum_{i=1}^{N} q_i(t) = 1.
\tag{6.15}
\end{equation}
定义投资组合 $h$ 为
\begin{equation}
h_i(t) = \frac{q_i(t)Z(t)}{S_i(t)}.
\tag{6.16}
\end{equation}

由此, 价值过程 $V^h$ 由 $V^h = Z$ 给出, 因此 $(h,\,c)$ 是自融资的, 并且其对应的相对投资组合 $u$ 由 $u = q$ 给出. 

\medskip
\noindent\textbf{证明}~
根据定义, 价值过程 $V^h$ 由 $V^h(t) = h(t)S(t)$ 给出, 因此由式 (6.15) 与 (6.16) 可得
\begin{equation}
V^h(t) = \sum_{i=1}^{N} h_i(t)S_i(t)
       = \sum_{i=1}^{N} q_i(t)Z(t)
       = Z(t)\sum_{i=1}^{N} q_i(t)
       = Z(t).
\tag{6.17}
\end{equation}

将式 (6.17) 代入 (6.16), 可得与 $h$ 对应的相对投资组合 $u$ 为 $u = q$.   
再将式 (6.17) 与 (6.16) 代入 (6.14), 可得
\[
dV^h(t) = \sum_{i=1}^{N} h_i(t)\,dS_i(t) - c(t)\,dt,
\]
这表明 $(h,\,c)$ 为自融资投资组合. \qed

\section{股息}

本节内容仅在第 16 章中会再次使用. 我们重新考虑第 6.1 节的设定与记号, 但现在假设资产可能支付股息. 

\medskip
\noindent\textbf{定义 6.6}~
设给定过程 $D_1(t), \ldots, D_N(t)$, 其中 $D_i(t)$ 表示在区间 $(0,\,t]$ 内, 每单位资产 $i$ 的持有者所获得的\textbf{累计股息}.   
若 $D_i$ 具有形式
\begin{equation}
dD_i(t) = \delta_i(t)\,dt,
\end{equation}
其中 $\delta_i$ 为某个过程, 则称资产 $i$ 具有\textbf{连续股息收益率}(continuous dividend yield). 

在区间 $(s,\,t]$ 内, 每单位资产 $i$ 的持有者所获股息为 $D_i(t) - D_i(s)$.   
若其具有股息收益率形式, 则有
\[
D_i(t) = \int_0^t \delta_i(s)\,ds.
\]

我们假设所有股息过程都具有随机微分形式. 

接下来推导自融资投资组合的动态. 与往常一样, 我们定义价值过程
\[
V(t) = h(t)S(t).
\]

当前情形与无股息情形的差异在于:预算方程 (6.2) 需要进行修正. 

在当前时刻 $t$, 我们所能支配的资金包括两个部分:

\begin{itemize}
    \item 旧投资组合的价值, 照常为
    \[
    h(t-\Delta t)S(t). 
    \]
    \item 在区间 $(t-\Delta t,\,t]$ 内所获得的股息. 这部分为
    \[
    \sum_{i=1}^{N} h_i(t-\Delta t)\,[D_i(t) - D_i(t-\Delta t)] = h(t-\Delta t)\Delta D(t). 
    \]
\end{itemize}

因此, 相关的预算方程为
\begin{equation}
h(t-\Delta t)S(t) + h(t-\Delta t)\Delta D(t) = h(t)S(t) + c(t)\Delta t. 
\tag{6.18}
\end{equation}

按照与第 6.1 节相同的推理, 我们可得到自融资投资组合的动态为
\[
dV(t) = \sum_{i=1}^{N} h_i(t)\,dS_i(t)
       + \sum_{i=1}^{N} h_i(t)\,dD_i(t)
       - c(t)\,dt. 
\]

我们将其写为如下形式化定义. 

\medskip
\noindent\textbf{定义 6.7}~
\begin{enumerate}
    \item \textbf{价值过程} $V^h$ 定义为
    \begin{equation}
    V^h(t) = \sum_{i=1}^{N} h_i(t)\,S_i(t). 
    \tag{6.19}
    \end{equation}

    \item \textbf{(向量值)收益过程} $G$ 定义为
    \begin{equation}
    G(t) = S(t) + D(t). 
    \tag{6.20}
    \end{equation}

    \item 若投资组合–消费对 $(h,\,c)$ 满足
    \begin{equation}
    dV^h(t) = \sum_{i=1}^{N} h_i(t)\,dG_i(t) - c(t)\,dt, 
    \tag{6.21}
    \end{equation}
    则称其为\textbf{自融资的}. 
\end{enumerate}

\medskip
\noindent\textbf{引理 6.8}~
用相对权重表述时, 自融资投资组合的动态可以写为
\begin{equation}
dV^h(t) = V(t)\cdot \sum_{i=1}^{N} u_i(t)\frac{dG_i(t)}{S_i(t)} - c(t)\,dt.
\tag{6.22}
\end{equation}

\section{练习}

\noindent\textbf{练习 6.1}~
在带股息的情形下, 补全自融资投资组合动态推导的细节. 

\medskip
\noindent\textbf{解答 6.1}~
从股息情形的离散期预算约束出发,
\[
h(t-\Delta t)S(t) + h(t-\Delta t)\Delta D(t) = h(t)S(t) + c(t)\Delta t,
\]
其中 $\Delta D(t) = D(t) - D(t-\Delta t)$. 为得到伊藤型微分, 在左侧加减 $S(t-\Delta t)\Delta h(t)$ 并整理得
\[
S(t-\Delta t)\Delta h(t) + \Delta S(t)\Delta h(t) + \Delta D(t)\,h(t-\Delta t) + c(t)\Delta t = 0.
\]
令 $\Delta t \to 0$ 并按伊藤意义取极限, 得到
\[
S(t)\,dh(t) + dh(t)\,dS(t) + h(t)\,dD(t) + c(t)\,dt = 0.
\]
另一方面, 投资组合价值为 $V(t) = h(t)S(t)$, 其伊藤微分为
\[
dV(t) = h(t)\,dS(t) + S(t)\,dh(t) + dS(t)\,dh(t).
\]
将上一式中的 $S(t)\,dh(t) + dS(t)\,dh(t)$ 用预算式替换, 得
\[
dV(t) = \sum_{i=1}^{N} h_i(t)\,dS_i(t) + \sum_{i=1}^{N} h_i(t)\,dD_i(t) - c(t)\,dt.
\]
记 $G(t) = S(t) + D(t)$, 则 $dG_i(t) = dS_i(t) + dD_i(t)$, 因而
\[
dV(t) = \sum_{i=1}^{N} h_i(t)\,dG_i(t) - c(t)\,dt.
\]
再用相对权重 $u_i(t) = \dfrac{h_i(t)S_i(t)}{V(t)}$ 表示 $h_i(t) = \dfrac{u_i(t)V(t)}{S_i(t)}$, 代入上式得到
\[
dV(t) = V(t)\sum_{i=1}^{N} u_i(t)\frac{dG_i(t)}{S_i(t)} - c(t)\,dt,
\]
即为式 (6.22). \qed

% \begin{chapterexercises}
% \begin{exercise}
% (练习 6.1)
% \end{exercise}
% \begin{exercise}
% (练习 1.2)
% \end{exercise}
% \end{chapterexercises}

% \begin{chapternotes}
% (Notes to Chapter 1)
% \end{chapternotes}

% ========== Chapter 2 ==========
\chapter{套利定价}


\section{引言}

在本章中, 我们将研究前一章所建立的一般模型的一个特殊情形. 我们基本上遵循 Merton (1973) 的论证, 其仅需用到前几章所介绍的数学工具. 完整的讨论见第 10 章. 

设想一个金融市场仅包含两种资产: 一种为无风险资产, 其价格过程为 $B$;另一种为股票, 其价格过程为 $S$. 那么, 什么是无风险资产呢?

\medskip
\noindent\textbf{定义 7.1}~
若价格过程 $B$ 满足动态方程
\begin{equation}
dB(t) = r(t)B(t)\,dt,
\tag{7.1}
\end{equation}
则称 $B$ 为\textbf{无风险资产}的价格过程, 其中 $r$ 为任意适应过程. 

\medskip
无风险资产的定义性质是, 它的动态中不含驱动布朗运动项. 由此可得
\[
\frac{dB(t)}{dt} = r(t)B(t),
\]
因此 $B$ 过程可写为
\[
B(t) = B(0)\exp\!\left(\int_0^t r(s)\,ds\right).
\]

对无风险资产的自然解释是: 它对应于具有(可能是随机的)短期利率 $r$ 的银行账户. 一种重要的特例是 $r$ 为确定常数的情形, 此时我们可以将 $B$ 解释为债券价格. 

我们假设股票价格 $S$ 满足
\begin{equation}
dS(t) = S(t)\alpha(t, S(t))\,dt + S(t)\sigma(t, S(t))\,d\widetilde{W}(t),
\tag{7.2}
\end{equation}
其中 $\widetilde{W}$ 是 Wiener 过程, $\alpha$ 与 $\sigma$ 为给定的确定性函数.   
之所以使用 $\widetilde{W}$ 而非更简单的 $W$ 记号, 原因将在后文中说明. 函数 $\sigma$ 称为 $S$ 的\textbf{波动率}(volatility), 而 $\alpha$ 称为 $S$ 的\textbf{局部收益率}(local mean rate of return). 

\paragraph{注记 7.1.1}~
注意上文中所建模的风险资产价格 $S$ 与无风险资产 $B$ 的区别.   
$B$ 的收益率形式上由下式给出:
\[
\frac{dB(t)}{B(t)\,dt} = r(t). 
\]
该量在局部意义下是确定的(locally deterministic), 即在时刻 $t$, 我们仅需观察当前的短期利率 $r(t)$ 便可完全知晓收益.   
与之相比, 股票 $S$ 的收益率形式上为
\[
\frac{dS(t)}{S(t)\,dt} = \alpha(t, S(t)) + \sigma(t, S(t))\frac{d\widetilde{W}(t)}{dt}, 
\]
该量在时刻 $t$ 并不可观测. 它包含了两部分:$\alpha(t, S(t))$ 与 $\sigma(t, S(t))$ 均可在时刻 $t$ 观察到, 而“白噪声”项 $d\widetilde{W}(t)/dt$ 是随机的. 因此, 与无风险资产不同, 股票具有\textbf{随机收益率}(stochastic rate of return), 即使在无限小的时间尺度上也是如此. 

上述模型的一个重要特例是 $r$、$\alpha$ 与 $\sigma$ 均为确定常数的情形. 这便是著名的\textbf{Black–Scholes 模型}. 

\medskip
\noindent\textbf{定义 7.2}~
\textbf{Black–Scholes 模型}由两个资产组成, 其动态为
\begin{align}
dB(t) &= rB(t)\,dt, \tag{7.3}\\
dS(t) &= \alpha S(t)\,dt + \sigma S(t)\,d\widetilde{W}(t), \tag{7.4}
\end{align}
其中 $r$、$\alpha$ 与 $\sigma$ 均为确定常数. 

\section{或有权利与套利}

我们以由式 (7.1)–(7.2) 给出的金融市场模型为基础, 进入本书的核心问题——\textbf{金融衍生品定价}.   
稍后我们将给出严格的数学定义, 但首先介绍最重要的一种衍生品:\textbf{欧式看涨期权}. 

\medskip
\noindent\textbf{定义 7.3}~
设标的资产为 $S$, 执行价格(或行权价)为 $K$, 到期时间(或行权日)为 $T$, 则欧式看涨期权定义如下:

\begin{itemize}
    \item 期权持有人在时刻 $T$ 有权以价格 $K$ 购买一股标的资产 $S$;
    \item 期权持有人没有义务必须购买该资产;
    \item 以价格 $K$ 购买标的资产的权利只能在确切时刻 $T$ 行使. 
\end{itemize}

需要注意的是, 执行价格 $K$ 与到期时间 $T$ 均在期权合约签订时确定, 对我们而言通常是在 $t=0$.   
欧式看跌期权(European put option)与欧式看涨期权类似, 只是赋予持有人以预定执行价卖出标的资产的权利.   
美式期权(American call option)则允许持有人在到期之前的任意时刻行使买入标的资产的权利. 

所有这类合约的共同特点是:它们都完全以标的资产 $S$ 的价格来定义, 因此自然被称为\textbf{衍生品}或\textbf{或有权利}(derivative instruments or contingent claims). 下面我们给出或有权利的形式化定义. 

\medskip
\noindent\textbf{定义 7.4}~
设金融市场的价格过程为向量 $S$.   
具有到期时间(或行权时间)$T$ 的\textbf{或有权利}(contingent claim), 又称 $T$-claim, 是任意随机变量 $\mathcal{X} \in \mathcal{F}^S_T$.   
若 $\mathcal{X}$ 具有形式
\[
\mathcal{X} = \Phi(S(T)), 
\]
则称其为\textbf{简单或有权利}(simple claim). 函数 $\Phi$ 称为\textbf{合约函数}(contract function). 

\medskip
这一定义的含义是:或有权利是一种合约, 规定其持有人在到期时刻 $T$ 将获得随机金额 $\mathcal{X}$(可能为正也可能为负).   
要求 $\mathcal{X} \in \mathcal{F}^S_T$ 的含义是:在时刻 $T$ 时, 这笔支付的金额是可以确定的. 

我们可以看到, 欧式看涨期权即为一个简单的或有权利, 其合约函数为
\[
\Phi(x) = \max[x - K, 0]. 
\]

欧式看涨与看跌期权的合约函数图像可见于图 7.1–7.2.   
显然, 或有权利(例如欧式期权)本身就是一种金融资产, 因此它在市场上必然具有价格.   
该价格显然依赖于时间 $t$ 与标的资产价格 $S(t)$.   
我们的核心问题是确定这一合约的“公允”价格. 我们用标准记号表示为
\begin{equation}
\Pi(t;\,\mathcal{X}),
\tag{7.5}
\end{equation}
其中 $\Pi(t;\,\mathcal{X})$ 表示或有权利 $\mathcal{X}$ 的价格过程. 在简单或有权利的情形下, 我们有时也写作 $\Pi(t;\,\Phi)$. 

在 $t=T$ 的情形下, 情况最为简单. 以下以欧式看涨期权为例说明:

\begin{enumerate}
    \item 若 $S(T) \ge K$, 则通过行使期权购买一股标的资产可获得确定利润. 此时需支付 $K$ 单位货币. 
\end{enumerate}

\begin{figure}[htbp]
    \centering
    \includegraphics[width=0.75\textwidth]{figure/fig7_1.png}
    \caption{合约函数. 欧式看涨期权, $K = 100$}
    \label{fig:7_1}
\end{figure}

\begin{figure}[htbp]
    \centering
    \includegraphics[width=0.75\textwidth]{figure/fig7_2.png}
    \caption{合约函数. 欧式看跌期权, $K = 100$}
    \label{fig:7_2}
\end{figure}

然后, 我们立即在市场上以价格 $S(T)$ 卖出该资产, 从而获得净利润 $S(T) - K$ 单位货币. 

\begin{enumerate}
    \setcounter{enumi}{1}
    \item 若 $S(T) < K$, 则期权没有任何价值. 
\end{enumerate}

因此, 我们可以得出, 在时刻 $T$, 期权唯一合理的价格 $\Pi(T)$ 为
\begin{equation}
\Pi(T) = \max[S(T) - K, 0]. 
\tag{7.6}
\end{equation}

同理,对于更一般的或有权利 $\mathcal{X}$,我们有
\begin{equation}
\Pi(T; \mathcal{X}) = \mathcal{X},
\tag{7.7}
\end{equation}
而在简单或有权利的特例中,
\begin{equation}
\Pi(T; \mathcal{X}) = \Phi(S(T))。
\tag{7.8}
\end{equation}

然而,对于任意 $t < T$,要确定或有权利 $\mathcal{X}$ 的“正确”价格则并非显而易见。  
事实上,似乎根本不存在所谓“正确”或“公允”的价格。  
期权的价格(如同任何其他资产)当然是由市场(即期权市场)决定的,因此会受到多种因素的复杂影响,例如市场参与者的风险态度及其对未来股价的预期。  
因此,令人极为惊讶的事实是:在相当温和的假设下,存在一个公式(即 Black–Scholes 公式),它能给出期权的唯一价格。  

我们将作出的主要假设是市场在如下意义下是\textbf{有效的}:市场\textbf{不存在套利机会}(free of arbitrage possibilities)。  
下面给出这一核心概念的形式化定义。

\medskip
\noindent\textbf{定义 7.5}~
在一个金融市场上,若存在一个自融资投资组合 $h$ 满足
\begin{align}
V^h(0) &= 0, \tag{7.9}\\
P(V^h(T) \ge 0) &= 1, \tag{7.10}\\
P(V^h(T) > 0) &> 0, \tag{7.11}
\end{align}
则称市场上存在\textbf{套利机会}(arbitrage possibility)。

若不存在满足以上条件的 $h$,则称市场\textbf{无套利}(arbitrage free)。

\medskip
套利机会本质上意味着:无需承担任何风险,即可从无中生有地赚取正金额的情形。  
换言之,这相当于一台无风险赚钱机器,或者形象地说,是市场中的“免费午餐”。  
我们将套利机会解释为市场严重定价错误的情形,而我们的主要假设是市场是有效的,即不存在套利机会。

\medskip
\noindent\textbf{假设 7.2.1}~
我们假设价格过程 $\Pi(t)$ 满足:由 $(B(t), S(t), \Pi(t))$ 所构成的市场中不存在套利机会。

\medskip
一个自然的问题是:我们如何识别套利机会?  
一般而言,这个问题的完整答案需要相当复杂的概率工具,熟悉读者可参见第 10 章。  
幸运的是,这里一个部分性的结果已足以支撑我们后续的讨论。

\noindent\textbf{命题 7.6}~
假设存在一个自融资投资组合 $h$,其价值过程 $V^h$ 满足动态
\begin{equation}
dV^h(t) = k(t)V^h(t)\,dt,
\tag{7.12}
\end{equation}
其中 $k$ 为适应过程。那么必须有 $k(t) = r(t)$ 对所有 $t$ 成立,否则市场中存在套利机会。

\medskip
\noindent\textbf{证明}~
我们略述论证,并为简便起见,假设 $k$ 与 $r$ 为常数且 $k > r$。  
此时我们可以以利率 $r$ 向银行借款,并立即将这笔钱投入到以利率 $k$ 增长的投资组合 $h$ 中。  
因此在 $t=0$ 时净投资为零,而当 $t>0$ 时财富将为正。换言之,存在套利。  

反之,若 $r > k$,则我们做空投资组合 $h$ 并将所得资金存入银行,同样产生套利。  
对于非常数或随机的 $r$ 与 $k$,论证方式相同。 \qed

\medskip
上述结论的核心在于:若某投资组合的动态中不含布朗运动项,即为\textbf{局部无风险投资组合}(locally riskless portfolio),则该组合的收益率必须等于短期利率 $r$。  
换言之,存在一个 $k$ 为收益率的投资组合 $h$,在实际意义上等价于拥有短期利率 $k$ 的银行账户。  
我们可将该命题改述为:在无套利市场中,只能存在一个短期利率。

\medskip
现在我们回到或有权利 $\mathcal{X}$ 的价格过程 $\Pi(t; \mathcal{X})$ 的问题。  
其核心思想如下:由于该权利的定义完全依赖于标的资产的价格过程,我们应当能用标的资产的价格来表达其价格。若市场无套利,则这种定价方式应与标的资产的价格动态保持一致。

\medskip
举一个简单的例子:在无套利市场中,欧式看涨期权的价格必须满足
\[
\Pi(t) \le S(t)。
\]
因为若期权价格高于标的股票本身,则任何理性投资者都不会购买期权,而会直接购买股票。  

形式化地,假设在某时刻 $t$,有 $\Pi(t) > S(t)$。  
此时我们卖出一个期权,将部分资金用于购买标的股票,剩余部分存入银行(即购买无风险资产)。  
随后不再操作,直到时刻 $T$。  
到期时,我们需向期权持有人支付 $\max[S(T) - K, 0]$,但这笔钱可通过卖出所持股票获得。  
此时我们的最终财富为 $S(T) - \max[S(T) - K, 0]$,该值为正,再加上银行中获得的利息,总财富为正,因此构成套利。

因此显然,无套利市场的要求将对价格过程 $\Pi(t; \mathcal{X})$ 的行为施加某些约束。
% \begin{chapterexercises}
% \begin{exercise}
% (练习 2.1)
% \end{exercise}
% \begin{exercise}
% (练习 2.2)
% \end{exercise}
% \end{chapterexercises}

% \begin{chapternotes}
% (Notes to Chapter 2)
% \end{chapternotes}

% ========== Chapter 3 ==========
\chapter{The Fundamental Theorem of Asset Pricing}
\section{Martingales and Change of Measure}
\section{Equivalent Martingale Measures}
\section{Discounted Asset Prices}
\section{Completeness and Martingale Representation}
\section{Uniqueness of the Risk-Neutral Measure}
\section{Change of Numeraire}

\begin{chapterexercises}
\begin{exercise}
(练习 3.1)
\end{exercise}
\begin{exercise}
(练习 3.2)
\end{exercise}
\end{chapterexercises}

\begin{chapternotes}
(Notes to Chapter 3)
\end{chapternotes}

% ========== Chapter 4 ==========
\chapter{Interest Rate Models}
\section{Zero-Coupon Bonds}
\section{The Short Rate and the Money Account}
\section{The Vasicek and CIR Models}
\section{Forward Rate and Yield Curve}
\section{The Heath–Jarrow–Morton Framework}
\section{Change of Numeraire for Bonds}

\begin{chapterexercises}
\begin{exercise}
(练习 4.1)
\end{exercise}
\end{chapterexercises}

\begin{chapternotes}
(Notes to Chapter 4)
\end{chapternotes}

% ========== Chapter 5 ==========
\chapter{Stochastic Calculus Background}
\section{Wiener Process and Stochastic Integrals}
\section{Itô’s Lemma}
\section{Multi-dimensional Itô Formula}
\section{Martingales and Quadratic Variation}
\section{Stochastic Differential Equations}
\section{Numerical Schemes: Euler and Milstein}

\begin{chapterexercises}
\begin{exercise}
(练习 5.1)
\end{exercise}
\end{chapterexercises}

\begin{chapternotes}
(Notes to Chapter 5)
\end{chapternotes}

% ========== Chapter 6 ==========
\chapter{Optimal Portfolios and the CAPM}
\section{Mean–Variance Optimization}
\section{The Capital Market Line}
\section{The CAPM Relation}
\section{Continuous-Time Portfolio Optimization}
\section{The Merton Problem}
\section{Consumption and Investment}

\begin{chapterexercises}
\begin{exercise}
(练习 6.1)
\end{exercise}
\end{chapterexercises}

\begin{chapternotes}
(Notes to Chapter 6)
\end{chapternotes}

% ========== Chapter 7 ==========
\chapter{Incomplete Markets}
\section{Sources of Incompleteness}
\section{Equivalent Martingale Measures in Incomplete Markets}
\section{Utility-Based Pricing}
\section{Mean–Variance Hedging}
\section{Minimal Martingale Measure}
\section{Examples and Applications}

\begin{chapterexercises}
\begin{exercise}
(练习 7.1)
\end{exercise}
\end{chapterexercises}

\begin{chapternotes}
(Notes to Chapter 7)
\end{chapternotes}

% ========== Chapter 8 ==========
\chapter{Foreign Exchange and Numeraire Change}
\section{The FX Market and Forward Rates}
\section{Change of Numeraire Technique}
\section{Pricing Foreign Options}
\section{Quanto and Currency Options}
\section{Interest Rate Parity Revisited}

\begin{chapterexercises}
\begin{exercise}
(练习 8.1)
\end{exercise}
\end{chapterexercises}

\begin{chapternotes}
(Notes to Chapter 8)
\end{chapternotes}

% ========== Chapter 9 ==========
\chapter{Credit Risk and Defaultable Bonds}
\section{Reduced Form Models}
\section{Hazard Rate and Survival Probability}
\section{Pricing Defaultable Zero-Coupon Bonds}
\section{Credit Spreads}
\section{Structural Models of Default}
\section{Credit Derivatives}

\begin{chapterexercises}
\begin{exercise}
(练习 9.1)
\end{exercise}
\end{chapterexercises}

\begin{chapternotes}
(Notes to Chapter 9)
\end{chapternotes}

% ========== Chapter 10 ==========
\chapter{Appendices and Mathematical Tools}
\section{Measure and Integration}
\section{Probability Theory Basics}
\section{Conditional Expectation and Martingales}
\section{Stopping Times and Optional Sampling}
\section{Itô’s Calculus Summary}
\section{Useful Formulas}

\begin{chapternotes}
(General Mathematical Notes)
\end{chapternotes}

% ===== 索引与术语 =====
\printglossary[title=List of Symbols]
\printglossary[type=\acronymtype,title=Abbreviations]
\printindex

% ===== 参考文献 =====
\backmatter
\printbibliography
\nocite{*}

\end{document}